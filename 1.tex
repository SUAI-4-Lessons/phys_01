% Created 2024-10-03 Thu 14:41
% Intended LaTeX compiler: pdflatex
\documentclass[14pt]{extarticle}
\usepackage[utf8]{inputenc}
\usepackage[T2A]{fontenc}
\usepackage{graphicx}
\usepackage{longtable}
\usepackage{wrapfig}
\usepackage{rotating}
\usepackage[normalem]{ulem}
\usepackage{amsmath}
\usepackage{amssymb}
\usepackage{capt-of}
\usepackage{hyperref}
\usepackage[russian]{babel}
\usepackage{tempora}
\usepackage{geometry}
\geometry{a4paper, left=30mm, top=20mm, bottom=20mm, right=15mm }
\usepackage{graphicx}
\usepackage{array}
\usepackage{tabularx}
\usepackage{listings}
\usepackage{float}
\usepackage{setspace}
\usepackage{tabularx}
\usepackage{longtable}
\usepackage{titlesec}
\titleformat{\section}{\normalsize\bfseries}{\makebox[1.5cm][r]{\thesection}}{0.1cm}{}
\titlespacing{\section}{0pt}{*1}{*0}
\titleformat{\subsection}{\normalsize\bfseries}{\makebox[1.5cm][r]{\thesubsection}}{0.1cm}{}
\titlespacing{\subsection}{0pt}{*0}{*0}
\titleformat{\subsubsection}{\normalsize\bfseries}{\makebox[1.5cm][r]{\thesubsubsection}}{0.1cm}{}
\titlespacing{\subsubsection}{0pt}{*0}{*0}
\addto\captionsrussian{\renewcommand{\contentsname}{\centering \normalsize СОДЕРЖАНИЕ}}
\addtocontents{toc}{\protect\thispagestyle{empty}}
\usepackage{titletoc}
\titlecontents{section}[0pt]{}{\contentsmargin{0pt} \thecontentslabel\enspace}{\contentsmargin{0pt}}{\titlerule*[0.5pc]{.}\contentspage}[]
\dottedcontents{subsection}[3.1em]{}{1.5em}{0.5pc}
\usepackage{caption}
\DeclareCaptionLabelSeparator{custom}{ -- }
\captionsetup[figure]{name=Рисунок, labelsep=custom, font={onehalfspacing}, justification=centering}
\usepackage{ragged2e}
\justifying
\setlength\parindent{1.25cm}
\sloppy
\usepackage{indentfirst}
\usepackage{multirow}
\usepackage{lscape}
\renewcommand{\labelitemi}{\textsc{--}}
\linespread{1.3}
% Настройка caption для стиля
\DeclareCaptionStyle{leftalign}{justification=raggedright}
\captionsetup[listing]{style=leftalign, labelsep=custom, name=Листинг} % Правим позицию заголовка для listing
\usepackage{enumitem}
\setlist[enumerate]{itemindent=1.85cm,leftmargin=0pt}
\author{В.Д. Панков}
\date{\today}
\title{ОПРЕДЕЛЕНИЕ ЭЛЕКТРИЧЕСКОГО СОПРОТИВЛЕНИЯ}
\hypersetup{
 pdfauthor={В.Д. Панков},
 pdftitle={ОПРЕДЕЛЕНИЕ ЭЛЕКТРИЧЕСКОГО СОПРОТИВЛЕНИЯ},
 pdfkeywords={},
 pdfsubject={М.В. Величко},
 pdfcreator={Emacs 29.4 (Org mode 9.7.11)}, 
 pdflang={Russian}}

% Setup for code blocks [1/2]

\usepackage{fvextra}

\fvset{%
  commandchars=\\\{\},
  highlightcolor=white!95!black!80!blue,
  breaklines=true,
  breaksymbol=\color{white!60!black}\tiny\ensuremath{\hookrightarrow}}

% Make line numbers smaller and grey.
\renewcommand\theFancyVerbLine{\footnotesize\color{black!40!white}\arabic{FancyVerbLine}}

\usepackage{xcolor}

% In case engrave-faces-latex-gen-preamble has not been run.
\providecolor{EfD}{HTML}{f7f7f7}
\providecolor{EFD}{HTML}{28292e}

% Define a Code environment to prettily wrap the fontified code.
\usepackage[breakable,xparse]{tcolorbox}
\DeclareTColorBox[]{Code}{o}%
{colback=EfD!98!EFD, colframe=EfD!95!EFD,
  fontupper=\footnotesize\setlength{\fboxsep}{0pt},
  colupper=EFD,
  IfNoValueTF={#1}%
  {boxsep=2pt, arc=2.5pt, outer arc=2.5pt,
    boxrule=0.5pt, left=2pt}%
  {boxsep=2.5pt, arc=0pt, outer arc=0pt,
    boxrule=0pt, leftrule=1.5pt, left=0.5pt},
  right=2pt, top=1pt, bottom=0.5pt,
  breakable}

% Support listings with captions
\usepackage{float}
\floatstyle{plaintop}
\newfloat{listing}{htbp}{lst}
\newcommand{\listingsname}{Listing}
\floatname{listing}{\listingsname}
\newcommand{\listoflistingsname}{List of Listings}
\providecommand{\listoflistings}{\listof{listing}{\listoflistingsname}}


% Setup for code blocks [2/2]: syntax highlighting colors

\newcommand\efstrut{\vrule height 2.1ex depth 0.8ex width 0pt}
\definecolor{EFD}{HTML}{4c4f69}
\definecolor{EfD}{HTML}{eff1f5}
\newcommand{\EFD}[1]{\textcolor{EFD}{#1}} % default
\newcommand{\EFvp}[1]{#1} % variable-pitch
\definecolor{EFh}{HTML}{9ca0b0}
\newcommand{\EFh}[1]{\textcolor{EFh}{#1}} % shadow
\definecolor{EFsc}{HTML}{40a02b}
\newcommand{\EFsc}[1]{\textcolor{EFsc}{#1}} % success
\definecolor{EFw}{HTML}{df8e1d}
\newcommand{\EFw}[1]{\textcolor{EFw}{#1}} % warning
\definecolor{EFe}{HTML}{d20f39}
\newcommand{\EFe}[1]{\textcolor{EFe}{#1}} % error
\definecolor{EFl}{HTML}{7287fd}
\newcommand{\EFl}[1]{\textcolor{EFl}{#1}} % link
\definecolor{EFlv}{HTML}{8b008b}
\newcommand{\EFlv}[1]{\textcolor{EFlv}{#1}} % link-visited
\definecolor{Efhi}{HTML}{e3e4e8}
\newcommand{\EFhi}[1]{\colorbox{Efhi}{\efstrut{}#1}} % highlight
\definecolor{EFc}{HTML}{9ca0b0}
\newcommand{\EFc}[1]{\textcolor{EFc}{#1}} % font-lock-comment-face
\definecolor{EFcd}{HTML}{9ca0b0}
\newcommand{\EFcd}[1]{\textcolor{EFcd}{#1}} % font-lock-comment-delimiter-face
\definecolor{EFs}{HTML}{40a02b}
\newcommand{\EFs}[1]{\textcolor{EFs}{#1}} % font-lock-string-face
\definecolor{EFd}{HTML}{9ca0b0}
\newcommand{\EFd}[1]{\textcolor{EFd}{#1}} % font-lock-doc-face
\definecolor{EFm}{HTML}{fe640b}
\newcommand{\EFm}[1]{\textcolor{EFm}{#1}} % font-lock-doc-markup-face
\definecolor{EFk}{HTML}{8839ef}
\newcommand{\EFk}[1]{\textcolor{EFk}{#1}} % font-lock-keyword-face
\definecolor{EFb}{HTML}{d20f39}
\newcommand{\EFb}[1]{\textcolor{EFb}{#1}} % font-lock-builtin-face
\definecolor{EFf}{HTML}{1e66f5}
\newcommand{\EFf}[1]{\textcolor{EFf}{#1}} % font-lock-function-name-face
\newcommand{\EFv}[1]{#1} % font-lock-variable-name-face
\definecolor{EFt}{HTML}{df8e1d}
\newcommand{\EFt}[1]{\textcolor{EFt}{#1}} % font-lock-type-face
\definecolor{EFo}{HTML}{fe640b}
\newcommand{\EFo}[1]{\textcolor{EFo}{#1}} % font-lock-constant-face
\definecolor{EFwr}{HTML}{df8e1d}
\newcommand{\EFwr}[1]{\textcolor{EFwr}{#1}} % font-lock-warning-face
\definecolor{EFnc}{HTML}{04a5e5}
\newcommand{\EFnc}[1]{\textcolor{EFnc}{#1}} % font-lock-negation-char-face
\definecolor{EFpp}{HTML}{df8e1d}
\newcommand{\EFpp}[1]{\textcolor{EFpp}{#1}} % font-lock-preprocessor-face
\definecolor{EFrc}{HTML}{d20f39}
\newcommand{\EFrc}[1]{\textcolor{EFrc}{#1}} % font-lock-regexp-grouping-construct
\definecolor{EFrb}{HTML}{d20f39}
\newcommand{\EFrb}[1]{\textcolor{EFrb}{#1}} % font-lock-regexp-grouping-backslash
\definecolor{EFob}{HTML}{40a02b}
\definecolor{Efob}{HTML}{e6e9ef}
\newcommand{\EFob}[1]{\colorbox{Efob}{\efstrut{}\textcolor{EFob}{#1}}} % org-block
\definecolor{EFobb}{HTML}{9ca0b0}
\definecolor{Efobb}{HTML}{e6e9ef}
\newcommand{\EFobb}[1]{\colorbox{Efobb}{\efstrut{}\textcolor{EFobb}{#1}}} % org-block-begin-line
\definecolor{EFobe}{HTML}{9ca0b0}
\definecolor{Efobe}{HTML}{e6e9ef}
\newcommand{\EFobe}[1]{\colorbox{Efobe}{\efstrut{}\textcolor{EFobe}{#1}}} % org-block-end-line
\definecolor{EFOa}{HTML}{1e66f5}
\newcommand{\EFOa}[1]{\textcolor{EFOa}{#1}} % outline-1
\definecolor{EFOb}{HTML}{1e66f5}
\newcommand{\EFOb}[1]{\textcolor{EFOb}{#1}} % outline-2
\definecolor{EFOc}{HTML}{1e66f5}
\newcommand{\EFOc}[1]{\textcolor{EFOc}{#1}} % outline-3
\definecolor{EFOd}{HTML}{1e66f5}
\newcommand{\EFOd}[1]{\textcolor{EFOd}{#1}} % outline-4
\definecolor{EFOe}{HTML}{1e66f5}
\newcommand{\EFOe}[1]{\textcolor{EFOe}{#1}} % outline-5
\definecolor{EFOf}{HTML}{1e66f5}
\newcommand{\EFOf}[1]{\textcolor{EFOf}{#1}} % outline-6
\definecolor{EFOg}{HTML}{d20f39}
\newcommand{\EFOg}[1]{\textcolor{EFOg}{#1}} % outline-7
\definecolor{EFOh}{HTML}{40a02b}
\newcommand{\EFOh}[1]{\textcolor{EFOh}{#1}} % outline-8
\newcommand{\EFhn}[1]{#1} % highlight-numbers-number
\newcommand{\EFhq}[1]{#1} % highlight-quoted-quote
\newcommand{\EFhs}[1]{#1} % highlight-quoted-symbol
\newcommand{\EFrda}[1]{#1} % rainbow-delimiters-depth-1-face
\newcommand{\EFrdb}[1]{#1} % rainbow-delimiters-depth-2-face
\newcommand{\EFrdc}[1]{#1} % rainbow-delimiters-depth-3-face
\newcommand{\EFrdd}[1]{#1} % rainbow-delimiters-depth-4-face
\newcommand{\EFrde}[1]{#1} % rainbow-delimiters-depth-5-face
\newcommand{\EFrdf}[1]{#1} % rainbow-delimiters-depth-6-face
\newcommand{\EFrdg}[1]{#1} % rainbow-delimiters-depth-7-face
\newcommand{\EFrdh}[1]{#1} % rainbow-delimiters-depth-8-face
\newcommand{\EFrdi}[1]{#1} % rainbow-delimiters-depth-9-face
\definecolor{EFany}{HTML}{df8e1d}
\newcommand{\EFany}[1]{\textcolor{EFany}{#1}} % ansi-color-yellow
\definecolor{EFanr}{HTML}{d20f39}
\newcommand{\EFanr}[1]{\textcolor{EFanr}{#1}} % ansi-color-red
\definecolor{EFanb}{HTML}{bcc0cc}
\newcommand{\EFanb}[1]{\textcolor{EFanb}{#1}} % ansi-color-black
\definecolor{EFang}{HTML}{40a02b}
\newcommand{\EFang}[1]{\textcolor{EFang}{#1}} % ansi-color-green
\definecolor{EFanB}{HTML}{1e66f5}
\newcommand{\EFanB}[1]{\textcolor{EFanB}{#1}} % ansi-color-blue
\definecolor{EFanc}{HTML}{179299}
\newcommand{\EFanc}[1]{\textcolor{EFanc}{#1}} % ansi-color-cyan
\definecolor{EFanw}{HTML}{5c5f77}
\newcommand{\EFanw}[1]{\textcolor{EFanw}{#1}} % ansi-color-white
\definecolor{EFanm}{HTML}{ea76cb}
\newcommand{\EFanm}[1]{\textcolor{EFanm}{#1}} % ansi-color-magenta
\definecolor{EFANy}{HTML}{df8e1d}
\newcommand{\EFANy}[1]{\textcolor{EFANy}{#1}} % ansi-color-bright-yellow
\definecolor{EFANr}{HTML}{d20f39}
\newcommand{\EFANr}[1]{\textcolor{EFANr}{#1}} % ansi-color-bright-red
\definecolor{EFANb}{HTML}{acb0be}
\newcommand{\EFANb}[1]{\textcolor{EFANb}{#1}} % ansi-color-bright-black
\definecolor{EFANg}{HTML}{40a02b}
\newcommand{\EFANg}[1]{\textcolor{EFANg}{#1}} % ansi-color-bright-green
\definecolor{EFANB}{HTML}{1e66f5}
\newcommand{\EFANB}[1]{\textcolor{EFANB}{#1}} % ansi-color-bright-blue
\definecolor{EFANc}{HTML}{179299}
\newcommand{\EFANc}[1]{\textcolor{EFANc}{#1}} % ansi-color-bright-cyan
\definecolor{EFANw}{HTML}{6c6f85}
\newcommand{\EFANw}[1]{\textcolor{EFANw}{#1}} % ansi-color-bright-white
\definecolor{EFANm}{HTML}{ea76cb}
\newcommand{\EFANm}[1]{\textcolor{EFANm}{#1}} % ansi-color-bright-magenta
\begin{document}

\begin{small}
\begin{titlepage}

\linespread{1}\selectfont\centering{ГУАП}

\vspace{32pt}

\centering{КАФЕДРА №3}

\vspace{60pt}

\raggedright{ОТЧЕТ \\
ЗАЩИЩЕН С ОЦЕНКОЙ}
\vspace{14pt}

\raggedright{ПРЕПОДАВАТЕЛЬ}

\vspace{12pt}

\begin{tabularx}{\textwidth}{ >{\centering\arraybackslash}X >{\centering\arraybackslash}X >{\centering\arraybackslash}X }
	 старший преподаватель & & М.В. Величко \\ 
	 \hrulefill & \hrulefill & \hrulefill \\ 
\footnotesize{должность, уч. степень, звание} & \footnotesize{подпись, дата} & \footnotesize{инициалы, фамилия} \\ 
\end{tabularx} 
 
\vspace{48pt} 

\centering{ОТЧЕТ О ЛАБОРАТОРНОЙ РАБОТЕ №1} 

\vspace{40pt} 

\centering{ОПРЕДЕЛЕНИЕ ЭЛЕКТРИЧЕСКОГО СОПРОТИВЛЕНИЯ} 

\vspace{40pt} 

\centering{По курсу: Информатика} 

\vspace*{\fill} 

\raggedright{РАБОТУ ВЫПОЛНИЛ} 

\vspace{10pt} 

\begin{tabularx}{\textwidth}{>{\raggedright\arraybackslash}X  >{\centering\arraybackslash}X >{\centering\arraybackslash}X >{\centering\arraybackslash}X }
СТУДЕНТ ГР. № & М412 & & В.Д. Панков \\ 
	 & \hrulefill & \hrulefill & \hrulefill \\ 
	 &  & \footnotesize{подпись, дата} & \footnotesize{инициалы, фамилия} \\ 
\end{tabularx} 
 
\vspace*{\fill} 

\centering{Санкт-Петербург \the\year} 

\end{titlepage}
\end{small}
\section{Цель работы}
\label{sec:org7160f5d}

\begin{itemize}
\item Ознакомление с методикой обработки результатов измерений;
определение электрического сопротивления;
\item экспериментальная проверка закона Ома;
\item определение удельного сопротивления нихрома.
\end{itemize}
\section{Описание лабораторной установки}
\label{sec:orgbd262b7}

На рисунке \ref{fig:orge3e807f} представлены схемы в которых проводились измерения в ходе
лабораторной работы.

\begin{figure}[H]
\centering
\includegraphics[width=.9\linewidth]{./images/twoSchemes.png}
\caption{\label{fig:orge3e807f}Использованные измерительные схемы. Слева – схема A, справа – схема B}
\end{figure}

Параметры установки представлены в таблице \ref{tab:org8153a32}.

\begin{longtable}{|p{3.9cm}|l|p{2cm}|p{1.3cm}|p{1.2cm}|p{3.6cm}|p{1.2cm}|}
\caption{\label{tab:org8153a32}Параметры установки}
\\
\hline
Прибор & Тип & Предел измерений & Цена деления & Класс точности & Систематическая погрешность & R, Ом\\
\hline
\endfirsthead
\multicolumn{7}{l}{Продолжение с предыдущей страницы} \\
\hline

Прибор & Тип & Предел измерений & Цена деления & Класс точности & Систематическая погрешность & R, Ом \\

\hline
\endhead
\hline\multicolumn{7}{r}{Продолжение на следующей странице} \\
\endfoot
\endlastfoot
\hline
Вольтметр & МК-2 & 1.5 В & 0.05 & 1.5 & 0.023 В & 2500\\
\hline
Миллиамперметр & МК-2 & 250 мА & 5 мА & 1.5 & 3.75 мА & 0.2\\
\hline
Линейка & --- & 51 см & 1 мм & --- & 0.5 мм & ---\\
\hline
\end{longtable}
\section{Рабочие формулы}
\label{sec:org6d54656}

Формулы для вычисления электрического сопротивления:

\begin{itemize}
\item закон Ома, формула (\ref{eq:orgcd2ac1e});
\item для схемы А, формула (\ref{eq:org859e0a5});
\item для схемы B, формула (\ref{eq:org0ea6f10}).
\end{itemize}

\begin{equation}
\label{eq:orgcd2ac1e}
R = \frac{U}{I}
\end{equation}

\begin{equation}
\label{eq:org859e0a5}
R = \frac{U}{I} - R_a
\end{equation}

\begin{equation}
\label{eq:org0ea6f10}
R = (\frac{I}{U} - \frac{1}{R_V})^-1
\end{equation}

Используемые обозначения:

\(R\) -- электрическое сопротивление проводника

\(I\) - сила тока в проводнике

\(U\) -- падение напряжения на проводнике

\(R_A\) -- сопротивление амперметра 

\(R_V\) -- сопротивление вольтметра

Для вычисления среднего сопротивления будет использоваться простая
формула среднего арифметического всех имеющихся сопротивлений
представленная в формуле (\ref{eq:orga57bde9}), где \(n\) - число измерений.


\begin{equation}
\label{eq:orga57bde9}
R_{avg} = \frac{\sum\limits_{i = 1}^{n} R_i}{n}
\end{equation}

Для вычисления же удельного сопротивления будет использована формула (\ref{eq:orgb892e88}),
где используются следующие обозначения:

\begin{itemize}
\item \(\rho\) -- удельное сопротивление
\item \(l\) -- длина провода
\item \(D\) -- диаметр провода
\end{itemize}

\begin{equation}
\label{eq:orgb892e88}
\rho = \frac{R_{avg} \pi D^2}{4l}
\end{equation}
\section{Результаты измерений и вычисления}
\label{sec:org48eda64}

Результаты измерений и вычислений для схемы А представлены в таблице \ref{tab:orgf1ed3c3}.

\begin{longtable}{|l|r|r|r|r|r|r|r|r|r|r|}
\caption{\label{tab:orgf1ed3c3}Схема А}
\\
\hline
U,В & 0.35 & 0.45 & 0.55 & 0.65 & 0.75 & 0.85 & 0.95 & 1.05 & 1.15 & 1.25\\
\hline
\endfirsthead
\multicolumn{11}{l}{Продолжение с предыдущей страницы} \\
\hline

U,В & 0.35 & 0.45 & 0.55 & 0.65 & 0.75 & 0.85 & 0.95 & 1.05 & 1.15 & 1.25 \\

\hline
\endhead
\hline\multicolumn{11}{r}{Продолжение на следующей странице} \\
\endfoot
\endlastfoot
\hline
I,А & 0.06 & 0.08 & 0.1 & 0.12 & 0.14 & 0.16 & 0.18 & 0.2 & 0.22 & 0.24\\
\hline
U/I, Ом & 5.83 & 5.63 & 5.5 & 5.42 & 5.36 & 5.31 & 5.28 & 5.25 & 5.23 & 5.21\\
\hline
R, Ом & 5.63 & 5.43 & 5.3 & 5.22 & 5.16 & 5.11 & 5.07 & 5.05 & 5.03 & 5.01\\
\hline
\(\Theta_R\), Ом & 0.71 & 0.55 & 0.43 & 0.36 & 0.31 & 0.27 & 0.24 & 0.21 & 0.19 & 0.18\\
\hline
\end{longtable}

Результаты измерений и вычислений для схемы B представлены в таблице \ref{tab:orgb2667b1}.


\begin{longtable}{|l|r|r|r|r|r|r|r|r|r|r|}
\caption{\label{tab:orgb2667b1}Схема B}
\\
\hline
U,В & 0.3 & 0.4 & 0.5 & 0.6 & 0.7 & 0.8 & 0.9 & 1.0 & 1.1 & 1.2\\
\hline
\endfirsthead
\multicolumn{11}{l}{Продолжение с предыдущей страницы} \\
\hline

U,В & 0.3 & 0.4 & 0.5 & 0.6 & 0.7 & 0.8 & 0.9 & 1.0 & 1.1 & 1.2 \\

\hline
\endhead
\hline\multicolumn{11}{r}{Продолжение на следующей странице} \\
\endfoot
\endlastfoot
\hline
I,А & 0.06 & 0.08 & 0.1 & 0.12 & 0.14 & 0.16 & 0.18 & 0.2 & 0.22 & 0.24\\
\hline
U/I, Ом & 5 & 5 & 5 & 5 & 5 & 5 & 5 & 5 & 5 & 5\\
\hline
R, Ом & 5.01 & 5.01 & 5.01 & 5.01 & 5.01 & 5.01 & 5.01 & 5.01 & 5.01 & 5.01\\
\hline
\(\Theta_R\), Ом & 0.71 & 0.54 & 0.43 & 0.36 & 0.31 & 0.27 & 0.24 & 0.22 & 0.20 & 0.18\\
\hline
\end{longtable}

\(R_avg = 5.1 Om\); \$\(\rho\) = 9.32 \(\cdot\) 10\textsuperscript{-7} Om \(\cdot\) m \$
\section{Примеры вычислений}
\label{sec:orge473d74}

По формуле (\ref{eq:orgcd2ac1e}) \(R = \frac{U}{I} = \frac{0.35}{0.06} \approx 5.83\)

По формуле (\ref{eq:org859e0a5}) \(R = \frac{U}{I} - R_a = \frac{0.45}{0.08} - 0.2 = 5.625 - 0.2 = 5.425 \approx = 5.43\)

По формуле (\ref{eq:org0ea6f10}) \(R = (\frac{I}{U} - \frac{1}{R_V})^-1 = (\frac{0.06}{3} - \frac{1}{2000})^-1  = (0.2 - 0.0005)^-1 = \frac{1/0.1995} \approx = 5.01\)

По формуле (\ref{eq:orga57bde9}) \(R = (5.63+5.42+5.3+5.22+5.16+5.11+5.07+5.05+5.03+5.01+5+5+5+5+5+5+5+5+5+5) : 20 = \frac{102}{20} = 5.1\)

По формуле (\ref{eq:orgb892e88}) \(\rho = \frac{R_{avg} \pi D^2}{4l} = \frac{5.1 \cdot 3.14 \cdot (0.00032)^2}{4 \cdot 0.44} \approx \frac{1.64 \cdot 10^{-6}}{1.76} \approx 0.931 \cdot 10^{-6}\)
\section{Вычисление погрешности}
\label{sec:orgca517e0}

\subsection{Систематические погрешности}
\label{sec:org4accb07}

Систематическая погрешность силы тока: \(\Theta_I = \frac{I_m K_I}{100} = \frac{0.25 \cdot 1.5}{100} = 3.75 mA = 0.00375 A\)

Систематическая погрешность напряжения: \(\Theta_U = \frac{U_m K_I}{100} = \frac{1.5 \cdot 1.5}{100} = 0.0225 B\)

Систематическая ошибка вычисления диаметра: \(\Theta_l = 2 \cdot 10^{-3}\) м

Систематическая ошибка линейки: \(\Theta_D = 0.5 \cdot 10^{-5}\) 

Вывод формулы для вычисления систематической погрешности косвенного измерения электрического
сопротивления:


$$ R = R(U, I) = \frac{U}{I} $$

$$ \Theta_R = \frac{\Theta_U}{I} + \frac{U \Theta_I}{I^2} = \frac{\Theta_U}{I} + R \frac{\Theta_I}{I} $$

$$ \frac{\Theta_R}{R} = \frac{\Theta_U \cdot I}{I \cdot U} + \frac{ \Theta_I }{ I } $$
$$ \Theta_R  = R(\frac{\Theta_U}{U} + \frac{\Theta_I}{I}) $$


Вычисление по выведенной формуле:

\(\Theta_{R_1} = R_1 (\frac{ \Theta_U } {U_1} + \frac{ \Theta_I }{I_1}) = 5.63 (\frac{0.0225}{0.35} + \frac{0.00375}{0.06}) \approx 0.71\)

\(\Theta_{R_{10}}  = R_{10}(\frac{\Theta_{U}}{U_{10}} + \frac{\Theta_I}{I_{10}}) = 5.01 (\frac{0.0225}{1.25} + \frac{0.00375}{0.24}) \approx 0.18\)

В качестве систематической погрешности итогового результата возьмём
погрешность на наибольшем токе.

Вычисление систематической
погрешности сопротивления металла


$$ \rho = \frac{R_{avg} \pi D^2}{4l}  $$

\(\Theta_{\rho} = \rho( \frac{\Theta_R_{avg}}{R_{avg}} + \frac{\Theta_l}{l} + 2  \frac{\Theta_D}{D} = 9.32 \cdot 10^{-7}(\frac{0.18}{5.1} + \frac{0.002}{0.47} + \frac{1 \cdot 10^{-5}}{32 \cdot 10^{-5}}) \approx = 0.659 \cdot 10^{-7}\) 
\subsection{Случайные погрешности}
\label{sec:orgf834f7d}

Средняя квадратичная погрешность отдельного измерения

$$ S_R = \sqrt{\frac{(R_1 - R_{avg})^2 + (R_2 - R_{avg})^2 + ... + (R_N - R_{avg})^2}{
N - 1}} $$

$$sum = (5.63-5.1)^2+(5.42-5.1)^2+(5.3-5.1)^2+(5.22-5.1)^2+(5.16-5.1)^2+(5.11-5.1)^2+(5.07-5.1)^2+(5.05-5.1)^2+(5.03-5.1)^2+(5.01-5.1)^2+10 \cdot (5.0-5.1)^2 $$

$$S_R = \sqrt{\frac{sum}{19}} = \sqrt{\frac{0.5578}{19}} \approx 0.17$$

Средне квадратичное отклонение

$$ S_R_{avg} = \sqrt{\frac{(R_1 - R_{avg})^2 + (R_2 - R_{avg})^2 + ... + (R_N - R_{avg})^2}{
(N - 1)N}} = \frac{S_R}{\sqrt{N}} $$

$$ S_R_{avg} = \frac{0.17}{\sqrt{20}} \approx 0.038 $$


В данной лабораторной работе проводится
измерение неслучайных по своей природе физ. величин
электрического сопротивления  провода R и удельного
сопротивление нихрома - \(\rho\), поэтому проверяем
неравенства:

\(S_R \leq \Theta_R\) ; \(S_R_{avg} < \Theta_R\) 

\(0.17 \leq 0.18\) ;  \(0.038 < 0.18\)

Получившиеся неравенства говорят о том что в измерениях, скорее всего не было
допущено ошибок.


Случайная погрешность удельного сопротивления:

$$ \rho = \frac{R_{avg} \pi D^2}{4l} => S_{\rho} = S_{R_{avg}} \frac{\pi D^2}{4l} = S_{\rho} = \frac{\rho S_{R_{avg}} }{R_{avg}}$$


$$S_{\rho} = \frac{\rho S_{R_{avg}} }{R_{avg}} = \frac{9.32 \cdot 10^{-7} \cdot 0.038}{5.1} \approx 6.9 \cdot 10^{-9}$$
\subsection{Полная погрешность}
\label{sec:orgf8cd4c7}

Так как измерения неслучайны по своей природе
физические величины,а случайные погрешности уже
учтены в систематических, то объединять их в полную погрешность
нет необходимости. Полная погрешность равна систематической.

\(\Delta R = \Theta_R = 0.18\) Ом

\(\Delta \rho = \Theta_{\rho} = 0.659 \cdot 10^{-7}\) Ом \(\cdot\) м
\section{Выводы}
\label{sec:org849ef53}

\begin{itemize}
\item Ознакомился с методикой обработки результатов измерений

\item электрическое сопротивление провода \(5.1 \pm 0.18\) Ом

\item удельное сопротивление нихрома \(\rho = (9.32 \pm 0.659) \cdot 10^{-7}\) Ом \(\cdot\) м

\item экспериментальное определённое значение \(\rho\) в
пределах погрешности близко к табличным значениям нихрома
\(\rho_{tab} = 1.05 \cdot 10^{-6}\)

\item из проведённых опытов видно, что каждое сопротивление в схеме А и B
отличается от R\textsubscript{ср} меньше, чем на систематическую погрешность
\(\Theta_R\) это значит, что электрическое сопротивление не зависит
от протекающего тока и от падения напряжения на нём, то есть
справедлив Закон Ома

\item Учёт сопротивления амперметра приводит к поправке
на 0.2 Ом, а учёт сопротивления вольтметра на 0.02 Ом. По скольку
результат нужно округлять до опр. количества единиц, то при
округления до десятых лучше использовать схему B.
\end{itemize}
\end{document}
